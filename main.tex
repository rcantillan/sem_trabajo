
% Added link to preamble
\documentclass[english,12pt]{article}
\usepackage{hyperref}
\usepackage[utf8]{inputenc}
\usepackage{blindtext}
\usepackage{graphicx}
\usepackage{authblk}
%\usepackage{natbib}
\usepackage{xcolor}
\usepackage{float}
\usepackage{amsmath}
\usepackage[english]{babel}
\usepackage{graphicx} % Required for inserting images
\usepackage[paperheight=16cm,paperwidth=12cm,textwidth=8cm]{geometry}
\usepackage[flushleft]{threeparttable,booktabs}
\usepackage{tabulary,booktabs}
\usepackage{booktabs}
\usepackage{siunitx}
\usepackage{ragged2e}

\appto\TPTnoteSettings{\footnotesize}

\usepackage[
        backend=biber,
        style=authoryear-comp,
        sorting=nyt,
        style=apa
    ]{biblatex}
 \addbibresource{references.bib}


 \geometry{
 a4paper,
 total={170mm,257mm},
 left=30mm,
 right=30mm,
 top=30mm,
 bottom=30mm
 }

% Keywords command
\providecommand{\keywords}[1]
{
  \small	
  \textbf{\textit{Keywords---}} #1
}

\graphicspath{{output/}}


\addbibresource{}



\title{Trust and diversity as an outcome of associative behavior patterns in Chile}
\author{Roberto Cantillan$^{1}$, Gustavo Ahumada$^{2}$  \\
        \small $^{12}$Pontificia Universidad Católica de Chile \\
        %\small $^{2}$University B \\
}

\date{Junio 2023}


\begin{document}
\maketitle


\begin{abstract}
En línea con la tradición analítica de redes, buscamos profundizar en la conexión entre el comportamiento asociativo (o cívico asociativo) y resultados de capital social a nivel individual. De esta manera, nos enfocamos en las "membresías múltiples", entendidas como un mecanismo básico de constitución de un campo de relaciones interorganizacionales, y la emergencia de patrones estratificados en términos de composición y de alcance asociativo. Adicioanlmente, buscamos profundizar en la hipótesis de Putnam, la cual sugiere que las membresías asociativas se asocian positivamente con la confianza generalizada, sugiriendo que son los patrones más diversos de membresías (en efecto, aquellos que facilitan el acceso a círculos y posiciones sociales diversas) los cuales facilitan en desarrollo de capital social a nivel individual y colectivo (confianza generalizada, confianza vecinal y diversidad de contactos ocupacionales)
\end{abstract}
\hspace{10pt}

\keywords{Multiple memberships, Social capital, Trust, associative field} 

\newpage





\maketitle

\section{Introduction}

\section{Theory}

Voluntary associations provide an organized context for joint activities, so participation in organizations is likely to shape social networks \parencite{feld_focused_1981,mcpherson_social_1992}. In general, the analysis of social networks and associations emphasizes two aspects 1) Networks and associations are treated as contexts for socialization and the development of social cohesion. This perspective tends to emphasize mechanisms that facilitate the regulation and
establishment of norms, identities and generalized trust \parencite{glanville_voluntary_2004,glanville_why_2016,paxton_association_2007,paxton_trust_2018} - public good Networks and associations are considered sources of inequality, before which individuals develop strategies to access resources and positions of influence \parencite{bekkers_social_2008} - privileged-good view. The latter is the perspective adopted by this article.

\hspace{10pt}

The logic is that greater associative contact in instances of organizational solidarity not only generates greater affection and civic virtues in general, but also has a diversifying effect on the social networks of the actors and, therefore, increases the probability of
access to scarce resources \parencite{diani_social_1997,malinick_network_2013,tindall_network_2012,benton_uniters_2016}. In simple terms, it is plausible that participation in social movements fosters the development of social capital at the individual level, that is, in those who declare to be members or participate in social movements


\section{Methodology}

\section{Results}

\section{Discussion}

\section{Conclusion}
\newpage

\printbibliography

\end{document}
